\section{Introduction}
\label{sec:intro}

The bibliography section of every paper is of great importance, but
simultaneously it is an almost universal annoyance. Even when an author has a
good idea of the set of papers most related to their work, they still
have to go through the effort of hunting down the proper citations, playing an
awkward game of guess-the-conference with Google Scholar.  Some authors
solve this problem by constructing vast personal \bibtex repositories and
maintaining a rigid naming convention to avoid conflicts.  While this scheme can
be effective if one tends to work in a narrow area and re-uses most references
multiple times, it breaks down if one writes about a variety of topics, and of
course, it must be replicated by each new academic at the beginning of their
career.

Academic search engines such as \cite{CiteSeer}, \cite{Scholar},
\cite{MSAcademic} have greatly simplified both the preliminary work of
investigating an area of study, as well the hunting down of bibliography entries
once a paper has been written.  Unfortunately, these services fall down when it
comes to identifying a set of papers most related to a reference document -- it
is left up to user to identify ``interesting'' keywords and to manually search
for relevant papers, find or create the needed reference entry and to update
their \bibtex file.

 \name relieves most of the burden associated with citations, by automatically
generating an appropriate \bibtex file based on the content similarity against a
large document corpus.  While determining the exact set of references an author
desires is an almost impossible goal, in our testing \name consistently finds
almost all the desired references for a paper and yields few false positives.