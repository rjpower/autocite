\section{Problem Description}
The goal of \name is to take as input an example document (as text), and emit
a set of candidate references in the desired format (either \bibtex entries or 
URLs).

In order to build a candidate corpus and verify the effectiveness of the system,
we needed to create effective title and citation extraction techniques.  As 
these differ from those previously used in literature (\cite{citeseer}), we
describe our methods in detail below.

\subsection{Title Extraction}
We describe our simple but effective title extraction metric.  In
short our metric is motivated by using the following observations:

A title is:
\begin{itemize}
\item centered and near the top of the first page
\item in a more prominent font then other texta
\end{itemize}

We codified these rules into a scoring heuristic for text elements.  Each text
element on the first page is assigned a score based on the following formula:

\[
  TitleScore(T) = {FontSize(T) \over log(HorizontalDist(T)) * log(TopDist(T)) }
\]

In this formula, HorizontalDist and TopDistance are approximating the distance
of the text element from the optimal title location -- HorizontalDist
measures the X-distance from the center, and TopDistance 
the Y-distance from the point roughly 10\% from the top of the document.

To find the title, the top scoring element is found based on this formula.  This
element is then joined with any elements nearby in the document which share the
same font.

Suprisingly, given the wide variety of title formats, this technique yields 
very good results - empirically we observed less then 0.1\% error rates in 
title extraction, including on documents that other reference engines have
trouble with.

\subsection{Citation Extraction}
Despite relatively well specified citation formats for common conferences
\cite{acm}, citation parsing is a messy business.  Previous academic literature
has focused on using machine learning techniques \cite{citeseer} to handle extracting citation
details (title, authors and conference) from documents.  Rather then duplicate
these efforts, we investigated whether citation extraction could be handled using
traditional parsing techniques (e.g. BNF grammars).

Our experience in this regard has been mixed.  While the resultant parser 
descriptions were compact and relatively easy to understand, they were 
also error-prone and difficult to debug.  A simple error would result in
many missed or erronous parses, and the ambiguous nature of the problem
complicates debugging.

Describe grammars / process here... etc...
